%%%%%%%%%%%%%%%%%%%%%%%%%%%%%%%%%%%%%%%%%
% Beamer Presentation
% Edited for, and by @papanikge
% 05/11/2016 Athens-Patras, Greece
%%%%%%%%%%%%%%%%%%%%%%%%%%%%%%%%%%%%%%%%%

\PassOptionsToPackage{unicode}{hyperref}
\PassOptionsToPackage{naturalnames}{hyperref}
\PassOptionsToPackage{table}{xcolor}
\documentclass[greek]{beamer}

\mode<presentation> {
%\usetheme{default}
%\usetheme{AnnArbor}
%\usetheme{Antibes}
%\usetheme{Bergen}
%\usetheme{Berkeley}
%\usetheme{Berlin}
%\usetheme{Boadilla}
%\usetheme{CambridgeUS}
%\usetheme{Copenhagen}
%\usetheme{Darmstadt}
%\usetheme{Dresden}
%\usetheme{Frankfurt}
%\usetheme{Goettingen}
%\usetheme{Hannover}
%\usetheme{Ilmenau}
%\usetheme{JuanLesPins}
%\usetheme{Luebeck}
\usetheme{Madrid}
%\usetheme{Warsaw}

\usecolortheme{beaver}
%\usecolortheme{fly}
%\usecolortheme{wolverine}

\setbeamertemplate{navigation symbols}{}
}

\usepackage{kmath,kerkis} 
\usepackage{epstopdf}
\usepackage{graphicx}
\usepackage{babel} % φο γκρικ
\usepackage[utf8]{inputenc}
\usepackage{booktabs}

%--------------------------------------------
%	TITLE PAGE SETTING
%--------------------------------------------

\title[\textlatin{Partial Escape Analysis}]{Μελέτη και Αξιολόγηση Τεχνικών
Ανάλυσης Μερικής Διαφυγής και Αντικατάστασης Βαθμωτών για Στατική Βελτιστοποίηση
στον Μεταγλωττιστή \textlatin{PyPy}}

 % for the title page and the footline..!
\author{Παπανικολάου Γεώργιος}
\institute[Πα.Πα.]
{
Πανεπιστήμιο Πατρών \\
\medskip
\textit{\textlatin{papanikge@ceid.upatras.gr}} \\
Α.Μ.: 5044

\medskip
\underline{Πτυχιακή εργασία}
}

\date{\today}

%--------------------------------------------
%	BEGIN
%--------------------------------------------

\begin{document}

\begin{frame}
\titlepage
\end{frame}

\begin{frame}
\frametitle{Eπισκόπηση}
\tableofcontents % set with sections{}. They don't matter for other shit
\end{frame}

%--------------------------------------------
%	PRESENTATION SLIDES
%--------------------------------------------

\section{Γενικά}

\begin{frame}
\frametitle{Βελτιστοποίηση κώδικα}

\begin{block}{\textlatin{Code optimization}}
Η βελτιστοποίηση κώδικα (\textlatin{code optimization}) στοχεύει στο να
μειώσει το κόστος του κώδικα σε:

\begin{enumerate}
	\item χρόνο που απαιτείται για το \textlatin{run} του προγράμματος.
	\item μνήμη που απαιτείται για τις μεταβλητές του.
\end{enumerate}
\end{block}

\vspace{4mm}
\textbf{Είδη βελτιστοποίησης:}

\begin{itemize}
	\item \underline{Δυναμική βελτιστοποίηση} – πχ.:
    	\textlatin{\small{dead code elimination, loop unrolling}}
	\item \underline{Στατική βελτιστοποίηση} – πχ.:
    	\textlatin{escape analysis}
\end{itemize}

\end{frame}

%--------------------------------------------

\section{Ο μεταγλωττιστής}
\subsection{Τι είναι το \textlatin{PyPy}}

\begin{frame}
\frametitle{Τί είναι το \textlatin{PyPy}?}

\begin{block}{\textlatin{RPython}}
Το \textlatin{RPython} είναι ένα "\textlatin{compiler framework}"
(γραμμένο σε \textlatin{Python}) το οποίο ουσιαστικά ορίζει την γλώσσα
\textlatin{"Restricted Python" (RPython)} και όλο το οικοσύστημά της.
\end{block}

Το \textlatin {framework} μπορεί να χρησιμοποιηθεί για να γραφτούν σε αυτό
μεταγλωττιστές με πολλές δυνατότητες όπως:

\begin{itemize}
	\item \textlatin{Garbage collection}
	\item \textlatin{Just in time compilation} (\textit{\textlatin{JIT}})
\end{itemize}

\begin{block}{\textlatin{PyPy}}
Το \textlatin{PyPy} είναι απλά ένας μεταγλωττιστής (\textlatin{interpreter})
για την \textlatin{Python} γραμμένος σε \textlatin{RPython}.
\end{block}

\end{frame}

%------------------------------------------------

\subsection{\textlatin{RPython pipeline}}

\begin{frame}
\frametitle{\textlatin{RPython pipeline}}

\begin{columns}[c] 

\column{.4\textwidth} % Left

Στην εικόνα \ref{pipeline} φαίνεται η διαδικασία της μεταγλώττισης που
ακολουθείται στον μεταγλωττιστή σε στάδια· από το πρώτο, τού πηγαίου κώδικα σε
\textlatin{Python} μέχρι την παραγωγή και την μεταγλώττιση κώδικα \textlatin{C}.

\column{.6\textwidth} % Right
\begin{figure}
\label{pipeline}
\caption{\textlatin{RPython Translation Pipeline}}
\includegraphics[width=0.45\linewidth]{pipeline.png}
\end{figure}

\end{columns}
\end{frame}

%------------------------------------------------
% CENTRAL THEORY FOLLOWS
%------------------------------------------------

\section{Ανάλυση Διαφυγής}
\subsection{Κανονική}

\begin{frame}
\frametitle{\textit{Κανονική} Ανάλυση Διαφυγής}

\begin{itemize}
	\item Η μέθοδος αυτή δρα πάνω στις μεταβλητές.
	\item \textlatin{Non-Peephole Optimization.} Σκανάρει όλο το συντακτικό δέντρο.
	\item Δουλεύει \textlatin{\texttt{Block}}-\textlatin{wise}
	\item Αναλύσει το πρόγραμμα και προσδιορίζει αν όντως η μεταβλητή απαιτείται.
\end{itemize}

\begin{block}{Μέθοδος}
Η μέθοδος αυτή προσδιορίζει αν η μεταβλητή χρειάζεται έξω από το \textit{\textlatin{scope}}
της τρέχουσας συνάρτησης, αν δηλαδή "διαφεύγει".\\
\vspace{3mm}
Μία μεταβλητή μπορεί να διαφύγει άν:
\begin{enumerate}
	\item Επιστρέφεται αυτή καθ' αυτή.
   	\item Τοποθετείται η ίδια σε μια καθολική (\textit{\textlatin{global}}) μεταβλητή.
\end{enumerate}
\end{block}

\end{frame}

%------------------------------------------------

\subsection{Μερική}

\begin{frame}
\frametitle{\textit{Μερική} Ανάλυση Διαφυγής}

\begin{block}{Μερική Διαφυγή – \textlatin{Partial Escape Analysis}}
Η ανάλυση μερικής διαφυγής λαμβάνει υπόψιν της και τα διάφορα
\textit{\textlatin{branches}} της ροής εκτέλεσης του κώδικα.
(βλ. \texttt{\textlatin{if}})
\end{block}

\vspace{5mm}

\textbf{Ο Αλγόριθμος\cite{stadler2014partial}:}

\begin{itemize}
	\item \underline{Δημιουργεί} ένα \textit{εικονικό αντικείμενο} (\textlatin{Virtual Object}) κάθε
    	φορά που ανακαλύπτει μια καινούργια μεταβλητή, και την \underline{αφαιρεί} από τον κώδικα, αφού
        την \underline{αντικαταστήσει} με βαθμωτό.
	\item \underline{Ανανεώνει} το αντικείμενο αυτό κατά τα συμφραζόμενά του στο πρόγραμμα, έτσι ώστε να
    	ξέρει να κάνει σωστή αντικατάσταση βαθμωτών στα επόμενα σημεία.
    \item Παράλληλα \underline{ελέγχει:} Αν συναντήσει κάποια από τις παραπάνω περιπτώσεις επανατοποθετεί
    	την μεταβλητή.
    \item \underline{Διατηρεί} τα αντικείμενα αυτά σε ένα \textlatin{dictionary} για αποδοτική αποθήκευση και 			ιδανικό \textlatin{alias handling}.
\end{itemize}

\end{frame}

%------------------------------------------------

\section{Υλοποίηση}
\subsection{Τρόπος υλοποίησης}

\begin{frame}
\frametitle{Τρόπος υλοποίησης}

\begin{block}{Μέρη}
Κάθε γράφημα αποτελείται απο:
\begin{itemize}
	\item \texttt{\textlatin{Blocks}}
	\item \texttt{\textlatin{Links}}
\end{itemize}
τα οποία εμείς μεταβάλλουμε σύμφωνα με την μέθοδο.
\end{block}

\begin{itemize}
	\item Τα \texttt{\textlatin{Links}} ενώνουν τα \texttt{\textlatin{Blocks}} κατά τη
	ροή του κώδικα της συνάρτησης
	\item Τα \texttt{\textlatin{Blocks}} περιέχουν τις εντολές (\textlatin{operations}).\\
    (Υπάρχουν ειδικά \texttt{\textlatin{Blocks}} όπως π.χ. για έναρξη, \textlatin{exceptions} κλπ.)
\end{itemize}

\end{frame}

%------------------------------------------------

\begin{frame}
\frametitle{Δομή Αποθήκευσης Εικονικών Αντικειμένων}

\begin{figure}
\caption{\textlatin{P.E. (dictionary) Data Structure}}
\includegraphics[width=0.8\linewidth]{DS.png}
\end{figure}

\end{frame}

%------------------------------------------------

\subsection{Παράδειγμα}

\begin{frame}
\frametitle{Παράδειγμα Μετατροπής Γραφημάτων Συναρτήσεων}
Το παρακάτω είναι μια πολύ απλή συνάρτηση πρίν την εφαρμογή της
βελτιστοποίησης μερικής διαφυγής.
\begin{figure}
\caption{Ενα γράφημα του συστήματος γραφημάτων του μεταγλωττιστή
\textlatin{PyPy}}
\includegraphics[width=0.66\linewidth]{graph-bef.png}
\end{figure}
\end{frame}

\begin{frame}
\frametitle{Παράδειγμα Μετατροπής Γραφημάτων Συναρτήσεων}
\begin{columns}[c]
\column{.45\textwidth}
...και η ίδια συνάρτηση μετά την ανάλυση.
\column{.55\textwidth}
\begin{figure}
\caption{Βελτιστοποιημένο γράφημα}
\includegraphics[width=0.7\linewidth]{graph-after.png}
\end{figure}
\end{columns}
\end{frame}

%------------------------------------------------

\section{Αποτελέσματα}

\begin{frame}
\frametitle{Αποτελέσματα}

\begin{table}[h]
\centering
\caption{\textlatin{Time gains throughout all graphs}}
\label{table-time}
\begin{tabular}{lllll}

–– & \cellcolor[HTML]{C0C0C0}\textlatin{Total graphs/functions}
& \cellcolor[HTML]{C0C0C0}\textlatin{Graphs that changed}
& \cellcolor[HTML]{C0C0C0}\textlatin{Time gain} &  \\
\cellcolor[HTML]{C0C0C0}1o \textlatin{run} & \multicolumn{1}{c}{21320} & 
\multicolumn{1}{c}{2230} & \multicolumn{1}{c}{4.5\%} &  \\
\cellcolor[HTML]{C0C0C0}2o \textlatin{run} & \multicolumn{1}{c}{59433} & 
\multicolumn{1}{c}{4157} & \multicolumn{1}{c}{5.2\%} &  \\
 &  &  &  & 

\end{tabular}
\end{table}

\begin{table}[]
\centering
\caption{\textlatin{General median percentage of getfield
removal after escape analysis}}
\label{table-getfield}
\begin{tabular}{lllll}
–– & \cellcolor[HTML]{C0C0C0}\textlatin{Total graphs/functions}
& \cellcolor[HTML]{C0C0C0}\textlatin{Graphs that changed}
& \cellcolor[HTML]{C0C0C0}\textlatin{Getfield removal} &  \\
\cellcolor[HTML]{C0C0C0}1o \textlatin{run} & \multicolumn{1}{c}{21320}
& \multicolumn{1}{c}{2154} & \multicolumn{1}{c}{18\%} &  \\
\cellcolor[HTML]{C0C0C0}2o \textlatin{run} & \multicolumn{1}{c}{59433}
& \multicolumn{1}{c}{3281} & \multicolumn{1}{c}{7\%} &  \\
 &  &  &  & 
\end{tabular}
\end{table}
\end{frame}

%%%%%%%%%%%%%%%%%%%%%%%%%%%%%%%%%%%%%%%%%%%%%%%%%%%%%%%%%%%
% END OF CONTENT --- REFERENCES AND ACKNOWLEDGES FOLLOW
%%%%%%%%%%%%%%%%%%%%%%%%%%%%%%%%%%%%%%%%%%%%%%%%%%%%%%%%%%%

\begin{frame}
\frametitle{\latintext{References}}
\footnotesize
{
\begin{thebibliography}{99}
\latintext{
\bibitem[1]{stadler2014partial} Stadler, Lukas
and W{\"u}rthinger, Thomas and M{\"o}ssenb{\"o}ck, Hanspeter (2014)
\newblock Partial escape analysis and scalar replacement for Java
\newblock \emph{ACM} Proceedings of Annual IEEE/ACM International
Symposium on Code Generation and Optimization – Page 125.

\bibitem[2]{debray1995abstract} Debray, Saumya (1995)
\newblock Abstract interpretation and low-level code optimization
\newblock \emph{ACM} Proceedings of the 1995 ACM SIGPLAN symposium
on Partial evaluation and semantics-based program manipulation –
Pages 111-121

\bibitem[3]{pypy} PyPy Project website
\newblock \url{http://www.pypy.org}

\bibitem[4]{myfork} Thesis code on the mercurial fork
\newblock \url{http://bitbucket.org/papanikge/pypy}
}

\end{thebibliography}
}
\end{frame}

%------------------------------------------------

\begin{frame}
\Huge{\centerline{Τέλος}}
\vspace{10mm}
\normalsize{
Ευχαριστώ πολύ τους:
\begin{itemize}
\item κο. Γ. Γαροφαλάκη
\item κο. Αθ. Νικολακόπουλο
\item κο. \latintext{Carl Friedrich Bolz} - \latintext{PyPy Project}
\end{itemize}
}
\end{frame}

%----------------------------------------------------

\end{document}