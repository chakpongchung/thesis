%------------------------------------------------------------------------------

\chapter{Υλοποίηση}
\label{chapter4}

\section{Γενικά}


Σε αυτό το κεφάλαιο θα αναλύσουμε τις λεπτομέρειες περί της υλοποίησης και τις
δυσκολίες που προέκυψαν σε αυτή. Ο κυρίως κώδικας (δηλαδή το αρχείο
\texttt{partial\_escape.py}) δίνεται στο παράρτημα Α'. Οι γραμμές κώδικα που
εμφανίζονται στο παρακάτω κείμενο αναφέρονται σε αυτόν.


Όπως έχουμε ήδη αναφέρει ο κώδικας του project του PyPy χωρίζεται σε subsystems,
και modules. Το module που υλοποιούμε υπόκειται στο σύστημα RPython, στο
subsystem του translator, στην ομάδα των backend βελτιστοποιήσεων. Βρίσκεται
επομένως στο ακόλουθο μονοπάτι:
\path{pypy/rpython/translator/backendopt/partial_escape.py}. Ο κώδικας
βρίσκεται σε ένα fork\cite{fork} του επίσημου \textit{repository} του
PyPy\cite{repo} στη σελίδα bitbucket\footnote{\url{http://bitbucket.com}}. Η
υλοποίηση έγινε με την βοήθεια του εργαλείου διαχείρησης εκδόσεων κώδικα (SCM)
mercurial\cite{mercurial} το οποίο χρησιμοποιείται εσωτερικά από το project, και
φυσικά κάνουμε προσπάθειες να περιληφθεί στην επίσημη έκδοσή του.

%------------------------------------------------------------------------------

\section{Test-driven development}

Στην υλοποίηση χρησιμοποιήθηκε η μέθοδος ανάπτυξης λογισμικού βασισμένη σε tests
(\textit{test-driven development}\cite{tdd}). Είναι μια σχετικά καινούργια
μέθοδος ανάπτυξης που χρησιμοποιεί test κώδικα για να ελέγξει την ορθότητα του
κανονικού κώδικα των προγραμμάτων. Η μέθοδος αναπτύχθηκε φυσικά λόγω της μεγάλης
πολυπλοκότητας που έχουν τα διάφορα προγράμματα και συστήματα την σήμερον ημέρα.
Ο προγραμματιστής αφού κατανοήσει περίπου πως πρέπει να δομήσει το πρόγραμμα του
ξεκινά με την συγγραφή του test κώδικα. Αρχικά για τις απλές (και ίσως
τετριμένες) περιπτώσεις και σταδιακά αυξάνει την πολυπλοκότητα των tests και
φυσικά του κώδικα αντιστοίχως. Αφού γράψει ένα ή περισσότερα tests, συγγράφει το
πρόγραμμά του προσπαθώντας να κάνει τα tests να "τρέξουν" επιτυχώς. Η διαδικασία
συνεχίζεται φυσικά μέχρι να ολοκληρωθεί η διαδικασία ανάπτυξής ή να φτάσει ο
προγραμματιστής στην επιθυμητή πολυπλοκότητα. Με αυτόν τον τρόπο – αν προφανώς
όλα τα tests είναι επιτυχή – ο προγραμματιστής είναι σίγουρος ότι οι καινούργιες
αλλαγες που επέφερε στο project δεν επηρέασαν την ορθότητα του προγράμματος σε
κάποιο προηγούμενο στάδιο.

Το pypy χρησιμοποιεί αυτή την μέθοδο, οπότε και η υλοποίησή μας την ακολουθεί. Ο
test κώδικας του module που υλοποιήθηκε βρίσκεται στο ακόλουθο μονοπάτι:
\path{pypy/rpython/translator/backendopt/test/test_partial_escape.py}. Δεν
συμπεριλαμβάνεται στην εργασία αυτή για λόγους λακωνικότητας.

%------------------------------------------------------------------------------

\section{Δομή του κώδικα}

Ο κώδικας είναι σε ένα αρχείο. Χωρίζεται φυσικά σε συναρτήσεις. Η κυρίως
συνάρτηση είναι η \texttt{partial\_escape()}.

%------------------------------------------------------------------------------

\section{Γραφήματα}

general and a graph pic

\subsection{How to edit the graphs}

todo

%------------------------------------------------------------------------------

\section{Φάσεις σχεδιασμού}

todo

\subsection{Σειριακά}
\subsection{Split}
\subsection{Merge}
\subsection{Multi-merge}
\subsection{Loops}
\subsection{Function calling etc}

%------------------------------------------------------------------------------

\section{Γενικά για προβλήματα}

todo

%------------------------------------------------------------------------------
